\documentclass[12pt]{article}

\usepackage{tikz}
\usepackage[]{algorithm2e}
\usepackage{pst-node,pst-plot}
\usepackage{amsmath}
\usepackage{float}
\begin{document}
\title{Homework 7}
\author{Robbie McKinstry, Jack McQuown, Cyrus Ramavarapu}
\renewcommand{\today}{19 September 2016}
\renewcommand{\baselinestretch}{1.5}
\maketitle

\section*{Problem 8: }
Given an arbitrary tree $\mathcal{T}$ with both positive and negative
edge weights, finding the maximally weighted simple path, $\mathcal{P}$
can be found recursively by considering both the best linear path 
and the best kinked path available at a given vertex $v$.\\\\
Although this seems to suggest that naively recursing to the immediate
subproblem, in this case a sub tree $\mathcal{T'}$ of $\mathcal{T}$, and
extracting the two required values is sufficient, such thinking actually
proves to be problematic.  This is because although a path may not
be the immediate best choice, its inclusion into the solution may prove
beneficial at a later stage because it leads to an overall greater 
edge weight.\\\\
Accounting for the possibility that a path may not be immediately optimal,
gives rise to the following recursive algorithm that considers both the 
linear and kinked paths at a given vertex.\\
\begin{algorithm}[H]
\SetKw{Func}{Function:}
\SetKw{Inp}{Input:}
\SetKw{Glob}{Globals:}
\SetKw{Def}{Define:}
\SetKw{Ret}{Return:}
\Func{$MWSP$}\\
\Inp{$Tree\ \mathcal{T}$}\\
\uIf{$\mathcal{T}\ has\ no\ children$}
{$return\ 0$}
\Else{
$linear\ 1,\ linear\ 2 = max\_2(\forall\ children\ i:\ MWSP(i) + w_{pi})$\\
$best\ linear = max(linear\ 1, linear\ 2)$\\
$L.append(best\ linear)$\\
$kinked\ path = linear\ 1\ +\ linear\ 2$\\
$K.append(kinked\ path)$\\
\Ret{$L,\ K$}
}
\end{algorithm}
The maximum element from $L,\ K$ can be taken.  If this maximum value is less
than $0$, the maximum weighted simple path can be taken to be $0$ since not
not moving from any vertex will be optimal.\\\\
Although the recursive algorithm produces the maximum weight of a simple path
on a tree, an interative process can be developed that will run in linear
time by starting with the leaves and moving up the tree while keeping
track of the kinked and linear paths at every vertex.\\\\
\begin{algorithm}[H]
\SetKw{Func}{Function:}
\SetKw{Inp}{Input:}
\SetKw{Glob}{Globals:}
\SetKw{Def}{Define:}
\SetKw{Ret}{Return:}
\Func{$IT\ MWSP$}\\
\Inp{$Tree\ \mathcal{T}$}\\
\Glob{$A\ [\ ][\ ]$}\\
\ForEach{$Leaf\ Node\ v$}
{$A\ [v][0] = 0$\\
$A\ [v][1] = 0$}
\tcc{Assume the nodes are numbered from the leaves upward}
\For{$v\leftarrow first\ non-leaf$ to $n$}
{$linear\ 1,\ linear\ 2 = max\_2(\forall\ children\ i: A[v][i]+w_{vi})$\\
$A\ [v][0] = max(linear\ 1,\ linear\ 2)$\\
$A\ [v][1] = linear\ 1,\ linear\ 2$\\
}
\end{algorithm}
As with the recursive algorithm, the maximum value between the two rows of $A$
will be the answer, unless this value is negative.  In this case the answer
will be $0$ for the reason mentioned above.

\section*{Problem 9: }

\end{document}
