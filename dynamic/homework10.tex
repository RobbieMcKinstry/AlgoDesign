\documentclass[12pt]{article}

\usepackage{tikz}
\usepackage[ruled, vlined]{algorithm2e}
\usepackage{pst-node,pst-plot}
\usepackage{amsmath}
\usepackage{float}

\newcommand{\BigO}[1]{\ensuremath{\operatorname{\mathcal{O}}\bigl(#1\bigr)}}

\begin{document}
\title{Homework 9}
\author{Robbie McKinstry, Jack McQuown, Cyrus Ramavarapu}
\renewcommand{\today}{22 September 2016}
\renewcommand{\baselinestretch}{1.5}
\maketitle

\section*{Problem 10: }
\section*{Problem 11: }
\section*{Problem 17: }
To develop a dynamic programming algorithm to determine if the
set of gems $\mathcal{G}$ can be partitioned such that the two
partitions $\mathcal{P}$ and $\mathcal{Q}$ have the same value
and the same number of rubies and emeralds all possible partitions
can be initially considered.  Let the total value of all the gems
be $\mathcal{L}$ and the total number of gems be $n$.\\\\
In considering every possible partition, a new gem is added to 
either $\mathcal{P}$ or $\mathcal{Q}$.  This gem, labeled
$g_i$, will have a value $v_i$. Additionally, $g_i$ will be 
either an \textit{emerald} or a \textit{ruby}.  Since the value 
between the two partitions has to be equal and the quantity of the
gem type also has to be the same, these values need to be tracked.\\\\
Tracking these values as the tree of all possible partitions is
generated gives the following pruing rules.\\\\
\begin{enumerate}
    \item If the value of either partition becomes greater than $\mathcal{L}/2$
          the tree can be pruned.  This is because if a partition has a value
          greater than half of the max value, the other partition can never
          catch up.
    \item If the number of emeralds or rubies in either partition becomes
          greater than $n$, the tree can be pruned. This is because the other
          partition will never be able to get enough gems of the necessary
          type to restablish balance.
    \item If two branches have the same value, same number of rubies, and
          the same number of emeralds in both partitions, one branch can be
          arbitrarily pruned.
\end{enumerate}
Using these pruning rules gives the following polynomial time algorithm.   
\begin{algorithm}[H]
\SetKw{Func}{Function:}
\SetKw{Inp}{Input:}
\SetKw{Glob}{Globals:}
\SetKw{Def}{Define:}
\SetKw{Ret}{Return:}
\Func{$Gem\ Partition$}\\
\Inp{$\mathcal{G}$}\\
\Glob{$A[\ ][\ ][\ ][\ ][\ ][\ ][\ ][\ ]$}\\
\tcc{iterate through possible gems}
\For{$l=1$ to $n$}{
    \tcc{iterate through values for $\mathcal{P}$}
    \For{$h=1$ to $\mathcal{L}/2$}{
        \tcc{iterate through rubies for $\mathcal{P}$}
        \For{$j=1$ to $n/2$}{
            \tcc{iterate through emeralds for $\mathcal{P}$}
            \For{$k$ to $n/2$}{
                \tcc{iterate through values for $\mathcal{Q}$}
                \For{$x=1$ to $\mathcal{L}/2$}{
                    \tcc{iterate through rubies for $\mathcal{Q}$}
                    \For{$y=1$ to $n/2$}{
                        \tcc{iterate through emeralds for $\mathcal{Q}$}
                        \For{$z$ to $n/2$}{
                            \If{$A[l,h,j,k,x,y,z]$ is defined}{
                                \uIf{$g_l$ is a ruby}{
                                   $A[l+1,h+v_l,j+1,k,x,y,z]=1$\\ 
                                   $A[l+1,h,j,k,x+v_l,y+1,z]=1$\\
                                }
                                \Else{
                                   $A[l+1,h+v_l,j,k+1,x,y,z]=1$\\ 
                                   $A[l+1,h,j,k,x+v_l,y,z+1]=1$\\ 
                                }
                            }
                        }
                    }
                }
            }                    
        } 
    }    
}
\tcc{check if partition is possible}
\For{$i=1$ to $\mathcal{L}/2$}{
    \For{$j=1$ to $n/2$}{
        \For{$k=1$ to $n/2$}{
            \If{$A[n,i,j,k,i,j,k]$ is defined}{
                \If{$A[n,i,j,k,i,j,k] == 1$}{
                    \Ret{1}
                }
            }
        }
    }
}
\Ret{0}
\end{algorithm}
The answer to if a partition is possible will be found by searching the space
at the $n^{th}$ level where the values for partitions $\mathcal{P}$ and $\mathcal{Q}$
are the same while also having the same number of rubies and emeralds.\\\\
This algorithm runs in \BigO{n^5\mathcal{L}^2} time.  This is a $poly(n+\mathcal{L})$
if $poly(x)=x^7$.
\end{document}
