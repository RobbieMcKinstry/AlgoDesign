\documentclass[12pt]{article}

\usepackage{tikz}
\usepackage[ruled, vlined]{algorithm2e}
\usepackage{pst-node,pst-plot}
\usepackage{amsmath}
\usepackage{float}
\begin{document}
\title{Homework 9}
\author{Robbie McKinstry, Jack McQuown, Cyrus Ramavarapu}
\renewcommand{\today}{22 September 2016}
\renewcommand{\baselinestretch}{1.5}
\maketitle

\section*{Problem 10: }
\section*{Problem 11: }

\subsection{Part A:}

Given the final array, backtracking to identify which items were added to the knapsack is fairly straight-forward. The value $V[n, S]$ is the maximum of the values $Value[n-1, S]$ and $Value[n-1, S-w(n)] + v(n)$. To determine which of those two terms was selected as the final value, look at those two cells in the table and determine which is larger: $Value[n-1, S]$ or $Value[n-1, S-w(n)] + v(n)$. 

If the former is larger, then the nth item added was not added to the knapsack. Repeat this process starting at the cell $Value[n-1, S]$ until you terminate at the beginning of the array. 

If the latter is larger, then the nth item was added to the knapsack. Repeat this process starting at the cell $Value[n-1, S-w(n)] + v(n)$ until you terminate at the beginning of the array.

\subsection{Part B:}

To solve the Knapsack Problem using only O(L) space and in time O(nL), create an array $A$ of length $L$. The $i$th element in $A$ represents the maximum value available with a weight constraint $i$. Thus, the final element represents the solution. Populate the $i$th element in $A$ by taking the larger of the $A[i-1]$ and $A[i-w(i)] + v(i)$. The first of those two values is the maximum value when you do not add the $i$th element to the array (it is, of course, the maximum value of the previous weight). The second value is the maximum value available to the knapsack at weight $i$ if you definitely add the $i$th element to the knapsack.

\section*{Problem 17: }
\end{document}
