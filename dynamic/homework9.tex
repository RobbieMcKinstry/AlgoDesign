\documentclass[12pt]{article}

\usepackage{tikz}
\usepackage[ruled, vlined]{algorithm2e}
\usepackage{pst-node,pst-plot}
\usepackage{amsmath}
\usepackage{float}
\begin{document}
\title{Homework 8}
\author{Robbie McKinstry, Jack McQuown, Cyrus Ramavarapu}
\renewcommand{\today}{21 September 2016}
\renewcommand{\baselinestretch}{1.5}
\maketitle

\section*{Problem 13: }
\section*{Problem 14: }
The goal of this problem to to find a 
solution to the polynomial \[\left(\sum_{i=1}^{n}x_iv_i\right)\bmod n=\ L\bmod n\]
given a series of positive values for $v$ and $L$.  Also, $x$
can be $0$ or $1$.\\\\  
This problem is similar to the subset sum problem, except that
values larger than $L$ can not be immediately pruned.
Instead, at a given level, if two nodes have the same value 
$modulo\ n$, one can be arbitrarily pruned.\\\\
Additionally, since the goal is to find a solution to the
polynomial, a bit string can be created to consider the
possibilities in which a value was included or discluded in
the sum.\\\\
Enumerating all possibilities and then considering the 
above pruning rule gives way to the following algorithm.\\\\
\begin{algorithm}[H]
\SetKw{Func}{Function:}
\SetKw{Inp}{Input:}
\SetKw{Glob}{Globals:}
\SetKw{Def}{Define:}
\SetKw{Ret}{Return:}
\Func{$Modular\ Subset\ Sum$}\\
\Inp{$Positive\ Integers\ v_1,\dots,v_n, L$}\\
\For{$i$ to $n$}
{\For{$s$ to $L\bmod n$}
{\If{A[k,s] is defined}
{
\tcc{$::$ is the concatentation operator.  $a::c \rightarrow ac$}
   $A[k+1,s] = A[k,s]::0$\\
   $A[k+1, (s + v_{k+1}\bmod n)\bmod n] = A[k,s]::1$\\
}
}
}
\Ret{$A[n,L\bmod n]$}
\end{algorithm}
When this algorithm completes, the answer, if it exists,
will be the bit string at $A[n,L\mod n]$.  This bit string
starting, read from left to right, will represent the 
coefficient of each $v$ in the polynomial.
\section*{Problem 15: }
\section*{Problem 16: }
\end{document}
