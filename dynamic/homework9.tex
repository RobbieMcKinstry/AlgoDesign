\documentclass[12pt]{article}

\usepackage{tikz}
\usepackage[ruled, vlined]{algorithm2e}
\usepackage{pst-node,pst-plot}
\usepackage{amsmath}
\usepackage{float}
\begin{document}
\title{Homework 8}
\author{Robbie McKinstry, Jack McQuown, Cyrus Ramavarapu}
\renewcommand{\today}{21 September 2016}
\renewcommand{\baselinestretch}{1.5}
\maketitle

\section*{Problem 13: }
The algorithm in order to solve this problem first requires pruning rules in order to decrease the size of the binary tree that is generated from the subsets of {$v_1$},{\ldots},{$v_n$}.\\\\
The two pruning rules are:\\\\
1) If the absolute value of a node v is greater than (L/2), then we can prune that subtree rooted at the node v.\\
2) If two nodes have the same value and are at the same depth in the binary tree, then we can arbitrarily select one of the nodes and prune the subtree rooted at that node.\\\\
Below is a line representation of the valid area that a node can occupy in terms of the value it holds.\\
\begin{picture}(100,100)
\put(24,60){$-L/2$}
\put(150,60){$L/2$}
\put(97,30){$0$}
\multiput(10,50)(10,0){19}{\line(0,1){5}}
\thicklines
\put(40,50){\circle{3}}
\put(160,50){\circle{3}}
\put(0,50){\vector(1,0){200}}
\put (0,50){\vector(-1,0){1}}
\linethickness{2pt}
\put(40,50){\line(1,0){120}}
\end{picture}\\\\
Now we will derive an iterative, array-based algorithm.\\
Let A[k, S] hold the bitstring of the subset k with sum S, where a 1 in the bitstring indicates {$x_i$} = 1 and a 0 indicates {$x_i$} = 0. We compute A[k, S]: \\\\
\begin{algorithm}
\SetKw{Func}{Function:}
\SetKw{Inp}{Input:}
\Func{Sum\\}
\Inp {$int\ v_1$, $\ldots$, $v_n$}\\
\For{$k\leftarrow 1$ to $n$}
{\For{$S\leftarrow 0$ to $L$}
{$A[k][S] = A[k][S] :: 1$\\
{$A[k][S] = A[k][S - v_k] :: 0$}
}
}
{return $A[k][0]$}
\end{algorithm}\\
At A[k][0], the bitstring is stored for the subset k with a sum of 0.
\section*{Problem 14: }
The goal of this problem to to find a 
solution to the polynomial \[\left(\sum_{i=1}^{n}x_iv_i\right)\bmod n=\ L\bmod n\]
given a series of positive values for $v$ and $L$.  Also, $x$
can be $0$ or $1$.\\\\  
This problem is similar to the subset sum problem, except that
values larger than $L$ can not be immediately pruned.
Instead, at a given level, if two nodes have the same value 
$modulo\ n$, one can be arbitrarily pruned.\\\\
Additionally, since the goal is to find a solution to the
polynomial, a bit string can be created to consider the
possibilities in which a value was included or discluded in
the sum.\\\\
Enumerating all possibilities and then considering the 
above pruning rule gives way to the following algorithm.\\\\
\begin{algorithm}[H]
\SetKw{Func}{Function:}
\SetKw{Inp}{Input:}
\SetKw{Glob}{Globals:}
\SetKw{Def}{Define:}
\SetKw{Ret}{Return:}
\Func{$Modular\ Subset\ Sum$}\\
\Inp{$Positive\ Integers\ v_1,\dots,v_n, L$}\\
\For{$k = 0$ to $n$}
{\For{$s = 0$ to $L\bmod n$}
{\If{A[k,s] is defined}
{
\tcc{$::$ is the concatentation operator.  $a::c \rightarrow ac$}
   $A[k+1,s] = A[k,s]::0$\\
   $A[k+1, (s + v_{k+1}\bmod n)\bmod n] = A[k,s]::1$\\
}
}
}
\Ret{$A[n,L\bmod n]$}
\end{algorithm}
When this algorithm completes, the answer, if it exists,
will be the bit string at $A[n,L\mod n]$.  This bit string
starting, read from left to right, will represent the 
coefficient of each $v$ in the polynomial.

\section*{Problem 15: }
This problem is very similar to the Knapsack problem from class, except that it can be conceptualized slightly differently: instead of having a single list of items where you can only take an item once with no repeats, now you have a class of identical items, which you can select from repeatedly. This additional complexity adds a single inner loop to our otherwise fairly standard computation. The new loop represents choosing an item one, twice, thrice, \textit{etc}. Below, $max$ takes the pairwise maximum of scalar values, while $maxElement$ returns the highest scalar value in the array argument. 

\begin{algorithm}[H]
\SetKw{Func}{Function:}
\SetKw{Inp}{Input:}
\SetKw{Glob}{Globals:}
\SetKw{Def}{Define:}
\SetKw{Ret}{Return:}
\Func{$Repeated\ Knapsack$} \\
\Inp{$Positive\ Integers\ v_1,\dots,v_n, w_1, \dots w_n, W$} \\
\Glob{A zeroed array $A[n][W]$}
\For{$k = 1$ to $n$}{
	\tcc{generate level k+1 from level k}
	\For{$s = 1$ to $W$}{
		\tcc{generate solutions for each of the weights}
		\If{$A[k, s]$ is defined]}{
			\For{$p = 1$ to $W$}{
				$A[k+1, s + pw_{k}] = max(A[k+1, s + pw_{k}], maxElement(A[k, s]) + pv_{k})$
			}
		}
	}
}
\Ret{$maxElement(A[n, W]$}
\end{algorithm}

\section*{Problem 16: }

For this problem, we maintain a tuple representing a subset's current values and it's potential to catch to the other sequence. Once a term is not able to catch up, we prune that node from the tree. Lastly, we represent the subset as a bitstring, where each bit represents the inclusion or exclusion of the corresponding set element. Note that the number of inner loop iterations sums to $L$ since our "catch up" factor, which is monotone decreasing, can cross no more than an $L$-length distance, which is obvious when the inner list is represented as a linked list instead of as an array.

\begin{algorithm}[H]
\SetKw{Func}{Function:}
\SetKw{Inp}{Input:}
\SetKw{Glob}{Globals:}
\SetKw{Def}{Define:}
\SetKw{Ret}{Return:}
\Func{$Repeated\ Knapsack$} \\
\Inp{$Positive\ Integers\ v_1,\dots,v_n$} \\
\Glob{A array A[n] holding linked lists}

Let $S = \sum_{i=1}^{n} v_{i}^{3}$ and $P = \prod_{i=1}^{n} v_{i}$

$A[1] \leftarrow  (0, 1, S, P, emptyString )$ 

\For{$i = 1$ to $n$}{
	\tcc{generate level i+1 from level i}
	\ForEach{$(sum, prod, Rsum, Rprod, str) \in A[i]$}{
		\If{$sum = prod$}{
			\Ret{str}
		}
		\If{$sum+RSum < prod$}{
			Continue
		}
		\If{$prod*Rprod < sum$}{
			Continue
		}
		$A[i+1]$.append(
			$(sum, prod, Rsum, Rprod, str :: 0 )$
		)
		$A[i+1]$.append(
			$(sum + v_{i}^{3}, prod*v_{i}, Rsum - v_{i}^{3}, Rprod / v_{i}, str :: 1 )$
		)
	}
}

\end{algorithm}

\end{document}