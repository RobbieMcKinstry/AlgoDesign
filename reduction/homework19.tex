\documentclass[12pt]{article}

\usepackage{tikz}
\usepackage[ruled, vlined]{algorithm2e}
\usepackage{pst-node,pst-plot}
\usepackage{amsmath}
\usepackage{float}

\newcommand{\BigO}[1]{\ensuremath{\operatorname{\mathcal{O}}\bigl(#1\bigr)}}

\begin{document}
\title{Homework 19}
\author{Robbie McKinstry, Jack McQuown, Cyrus Ramavarapu}
\renewcommand{\today}{17 October 2016}
\renewcommand{\baselinestretch}{1.5}
\maketitle

\section*{Problem 17: }
The Fixed Hamiltonian path problem determines if an undirected graph
$\mathcal{G}$ has a simple path that spans all vertices in $\mathcal{G}$
between two given vertices $x$ and $y$.  If there exists a polynomial
time algorithm for this problem, it can be shown by reduction, that
there exists a polynomial time algorithm for the Hamiltonian Cycle
problem.\\\\
The Hamiltonian cycle problem takes an undirected graph $\mathcal{H}$
and determines if there is a simple cycle spanning all the vertices
of $\mathcal{H}$.  It is obvious that if the graph contains a vertex
with a neighborhood size $\leq 1$, there cannot be a cycle because this
vertex will either represent a deadend or an unreachable vertex.  
Accounting for this problem and iff it is assumed that the Fixed
Hamiltonian path problem can be solved in polynomial time, this
problem can also be solved in polynomial time as demonstrated
by the following algorithm.\\\\ 
\begin{algorithm}[H]
\SetKw{Func}{Function:}
\SetKw{Inp}{Input:}
\SetKw{Glob}{Globals:}
\SetKw{Ret}{Return:}
\Func{$HamCycle$}\\
\Inp{$\mathcal{G}$}\\
\For{$Vertex\ v\ in \mathcal{G}$}
{\If{$Neighborhood(v, \mathcal{G}\leq 1$}
{\Ret{$0$}}
}
\For{$Vertex\ v\ in\ \mathcal{G}$}
{
\tcc{Neighborhood(Vertex v, Graph G)}
\tcc{Returns the set of vertices adjacent to v in graph G}
    \For{$Vertex\ u\ in Neighborhood(v,\mathcal{G})$}
{\If{$FixedHamPath(v,u,\mathcal{G})==1$}
{\Ret{$1$}}
}
}
\Ret{$0$}
\end{algorithm}
Since it is assumed that the $FixedHamPath$ algorithm runs
in polynomial time, this algorithm also runs in polynomial time because
it only through all the vertices in the outerloop and in the innerloop
will loop through a maximum of $n-1$ vertices.  As a result, the
$FixedHamPath$ problem will be called \BigO{n^2} times.  Therefore
this algorithm will run in Poly-time.\\\\
Although the above algorithm demonstrates the reduction of the
Fixed Hamiltonian path problem from the Hamiltonian cycle problem, the
repeat calls to $FixedHamPath$ are inelegant. The reduction can be 
performed with a single call to $FixedHamPath$ by transforming the
input graph $\mathcal{G}$ in such a way that there will only be a
Fixed Hamiltonian Path if and only if there is also a Hamiltonian 
Cycle.  This can be accomplished by adding two disjoint vertices
to the original graph named $a$ and $b$.  Both $a$ and $b$ are then
made adjacent to two members in the neighborhood of an arbitrary node, $x$,
in the original graph $\mathcal{G}$.  The new graph, $\mathcal{H}$
will only have a Fixed Hamiltonian Path between $a$ and $b$
if there is a cycle that spans all the nodes in
the original graph.  This arises from the lack of a `start point' in
the cycle and the need to bridge $x$.
Pictorially this is represented by the following
sketch:\\\\\\\\\\\\\\\\\\   

Algorithmically, this is covered by the following code once again
assuming that there exists a polynomial time algorithm for the
Fixed Hamiltonian path problem and the abscence of any vertices with
a neighborhood $\leq 1$.\\\\ 
\begin{algorithm}[H]
\SetKw{Func}{Function:}
\SetKw{Inp}{Input:}
\SetKw{Glob}{Globals:}
\SetKw{Ret}{Return:}
\Func{$HamCycle$}\\
\Inp{$\mathcal{G}$}\\
\Begin{$\mathcal{H} = add\_disjoint\_vertex(a,\mathcal{G})$\\
{\If{$Neighborhood(v, \mathcal{G}\leq 1$}
{\Ret{$0$}}
}
$\mathcal{H} = add\_disjoint\_vertex(b,\mathcal{H})$\\
\tcc{$u$ is an arbitrary vertex in $\mathcal{H}$}
\tcc{$u+1$ is a member of neighborhood(u) in $\mathcal{H}$}
$\mathcal{H} = create\_edge(a,u+1,\mathcal{H})$\\
$\mathcal{H} = create\_edge(b,u,\mathcal{H})$\\
\Ret{$FixedHamPath(a,b,\mathcal{H})$}
}
\end{algorithm}
Due to the lack of explicit loops in this algorithm, it is obvious it
runs in polynomial time under the aforementioned assumptions.  However,
for the sake of clarity, vertices and edges can be added in Poly-time
depending on the datastructure used to store the graph.

\section*{Problem 18: }

Personally, we found that the Hint got us 90\% of the way through this problem, and that last 10\% did not require much creativity.
 The Hint did all of the heavy lifting.
 
 Given a graph $H$ and an integer $\ell$, we seek to solve vertex cover using an algorithm for the problem described, which we call \textit{MinimumConnection} or \textit{MinCon} for short. We show that vertex cover can be solved in poly time if \textit{MinCon} can be solved in polytime. Below is an high-level description of our algorithm.

First, let $G$ be a graph with a single vertex, \textit{Center}. Label each vertex in $H$ uniquely, and for each vertex in $H$ add a vertex in $G$ and connect that vertex to \textit{C}, maintaining the label from $H$ in $G$. This constructs $G$ such that a path exists from any vertex to another by passing through \textit{C}. Now, for each edge $e$ in $H$, add a vertex $v$ to $G$. Connect $v$ to another node $v'$ if and only if $e$ is an edge of the vertex with the label $v'$ in $H$. (Recall that the labels were copied over, so a vertex $v'$ in $G$ has a correspondingly vertex in $H$.) This results in each vertex added in this manner possessing two edges (since every edge in $H$ connects exactly two vertices). These edges are also connected to each other by pathing through \textit{Center}. 

Let \textit{Nodes} be the set of vertices in $G$ labeled as corresponding to a vertex in $H$ (and thus not an edge in $H$). Let \textit{count(edges)} be the number of edges in $H$. Finally, let $R = \{ x \in G\} - \textit{Nodes}$. Return the result of \textit{MinCon(G, R, 1+ $\ell + \textit{count(edges)}$)} 

\textit{MinCon} is satisfiable \textit{iff} there exists a vertex cover of size $\ell$ on $H$. An algorithm for \textit{MinCon} must add to $U$ all of the nodes in $R$, which is \textit{Center} and all of the vertices in $G$ representing edges in $H$ (that is, all of the vertices in $G$ not representing vertices in $H$). This leaves $\ell$ vertices left to be added to the set $U$, and only the vertices left in \textit{Nodes} to select from. Selecting a vertex $v$ from \textit{Nodes} creates a path connecting all of $v$'s edges to \textit{Center} since $v$ is connected to \textit{Center}, so all of the vertices in $R$ will be connected to each other if at least one of the vertices in \textit{Nodes} and also in $U$ has an edge to that vertex.

This algorithm operates in polynomial time (iff \textit{MinCon} operates in polynomial time). 

\end{document}
