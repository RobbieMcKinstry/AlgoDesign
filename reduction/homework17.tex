\documentclass[12pt]{article}

\usepackage{tikz}
\usepackage[ruled, vlined]{algorithm2e}
\usepackage{pst-node,pst-plot}
\usepackage{amsmath}
\usepackage{float}

\newcommand{\BigO}[1]{\ensuremath{\operatorname{\mathcal{O}}\bigl(#1\bigr)}}

\begin{document}
\title{Homework 17}
\author{Robbie McKinstry, Jack McQuown, Cyrus Ramavarapu}
\renewcommand{\today}{17 October 2016}
\renewcommand{\baselinestretch}{1.5}
\maketitle

\section*{Problem 9:}
The optimization version of the clique problem with returns the large clique in
undirected graph $\mathcal{G}$ can be shown to be self reducible in polynomial
time if the decision version of the clique problem has a polynomial time algorith.
Symbollically, this means the following.
\[
Clique_{opt} \leq_{poly} Clique_{eq}
\]
Since the optimization version of the clique problem returns the largest clique within
the graph $\mathcal{G}$, the following algorithm demonstrates the polynomial time
reduction under the assumption of a polynomial time algorithm for the decision problem.\\
\begin{algorithm}[H]
\SetKw{Func}{Function:}
\SetKw{Inp}{Input:}
\SetKw{Glob}{Globals:}
\SetKw{Ret}{Return:}
\Func{$Clique_{opt}$}\\
\Inp{$Graph\ \mathcal{G}$}\\
\tcc{$n$ is the number of vertices in graph G}
\tcc{this loop will eventually exit via return.  All graphs have a $1$ clique}
\While{$n\neq 0$}{
\uIf{$Clique_{dec}(\mathcal{G}, n)$}
{      
    \While{$Vert_{\mathcal{G}}\geq n$}{
        $v = Remove_Vertex(\mathcal{G},\ n)$\\
        \uIf{$Clique_{dec}(\mathcal{G}, n)$}
        {   continue }
        \Else{
            $mark\_vertex(v)$\\
            $Read\_Vertext{\mathcal{G},\ v}$
        }
    }
    \Ret{$\mathcal{G}_{verts}$} 
}
\Else{
    $n--$
}
}
\end{algorithm}
\section*{Problem 10:}
\section*{Problem 13:}
\section*{Problem 14:}
It can be shown by reduction that if there exists a polynomial time algorithm
for the \textit{4-coloring} problem there also exists a polynomial time algorithm
\textit{3-coloring}.  Mathematically this relationship can be expressed as follows.
\[
3\-coloring \leq_{poly} 4\-coloring
\]
The following algorithm demonstrates this reduction.\\
\begin{algorithm}[H]
\SetKw{Func}{Function:}
\SetKw{Inp}{Input:}
\SetKw{Glob}{Globals:}
\SetKw{Ret}{Return:}
\Func{$3-Coloring$}\\
\Inp{$Graph\ \mathcal{G}$}\\
\tcc{$n$ is the number of vertices in graph G}
$\mathcal{G'}=add\_connected\_vertex(\mathcal{G})$\\
\Ret{$4-coloring(\mathcal{G'})$}
\end{algorithm}
The reason this algorithm works is because if a graph had a \textit{3-coloring}, the
addition of a vertex connected to all vertices forces the addition of a new color because
if any of the three original colors are given to the new vertex, it will create a
violation of the coloring.  As a result, the new graph will have a \textit{4-coloring}.
In a similar manner, the addition of a new vertex to a graph that did not originally
have a \textit{3-coloring} will not result in a \textit{4-coloring}.\\\\
Lastly, the addition of a new vertex and the subsequent connecting of all the original
vertices to the new vertex can be done in polynomial time.  This is apparent if the 
original graph was stored in an adjacency matrix and a new row and column are added.
\end{document}
