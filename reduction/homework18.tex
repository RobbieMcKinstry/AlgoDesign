\documentclass[12pt]{article}

\usepackage{tikz}
\usepackage[ruled, vlined]{algorithm2e}
\usepackage{pst-node,pst-plot}
\usepackage{amsmath}
\usepackage{float}

\newcommand{\BigO}[1]{\ensuremath{\operatorname{\mathcal{O}}\bigl(#1\bigr)}}

\begin{document}
\title{Homework 18}
\author{Robbie McKinstry, Jack McQuown, Cyrus Ramavarapu}
\renewcommand{\today}{12 October 2016}
\renewcommand{\baselinestretch}{1.5}
\maketitle

\section*{Problem 15: }
\section*{Problem 16: }
The dominating set problem can be shown to be \textbf{NP-Hard} via 
a reduction to the vertex cover problem.  In the vertex cover problem,
the goal is to find a set of vertices of given cardinality \textit{k} 
that touches every edge.  For example, the following graph $\mathcal{G}$
has a vertex cover of $3$.\\\\
%Insert graph
The dominating set problem, on the otherhand, produces the minimum set 
of vertices, $\mathcal{S}$, from the input graph $\mathcal{H}$ such that every element
not in the dominating set $\mathcal{S}$ is adjacent to an element in the
dominating set $\mathcal{S}$.  This is demonstrated by the following example.\\\\
%Insert graph
The strategy to show vertex cover can be solved if there is a polynomial time 
algorithm for the dominating set problem will be local reduction.  Since in the
vertex cover problem there exists a minimum number of vertices that must be
included so that every edge is touched, this number of vertices can be related
to minimum number of vertices in the dominating set.  This can be accomplished by
forcing the dominating set problem to select one vertex for every edge.  This can
be done by looking at every edge and creating a triangle by adding a new vertex
and then adding an edge between the new vertex and the two original vertices.\\\\
Although this returns the minimum dominating set, this value will also represent
the minimum vertex cover.  Although the input to the vertex cover problem can be
different than this minimum value, the decision for this problem can be solved
by a comparison to check if the input value is greater than or less than the
result from the dominating set problem.  If it is greater than, the vertex cover
problem will return true; otherwise, it will return false.\\\\
%Insert graph
Algorithmically, this process is represented by the following code.\\
\begin{algorithm}[H]
\SetKw{Func}{Function:}
\SetKw{Inp}{Input:}
\SetKw{Glob}{Globals:}
\SetKw{Ret}{Return:}
\Func{Vertex Cover}\\
\Inp{Graph G, int k}\\
\tcc{$V_G$ is the number of vertices in graph G}
\For{edge $e$ in $\mathcal{G}$}
{
    $\mathcal{G'}=add\_triangle(\mathcal{G},\ e)$\\
}
$\mathcal{S}=dominating\_set(\mathcal{G'})$\\
\uIf{$|\mathcal{S}| \leq k$}
{
    \Ret{$1$}
}
\Else
{
    \Ret{$0$}
}
\end{algorithm}
A runtime analysis of this code shows that the triangles can be made in polynomial
time since the process only iterates over all the vertices. Furthermore, the 
necessary and sufficient condition of the transformation is satisfied by the 
existence of a triangle for every edge which forces the dominating set 
problem to pick at least one vertex in each triangle.  
\end{document}
