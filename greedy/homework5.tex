\documentclass[12pt]{article}

\usepackage[]{algorithm2e}
\usepackage{amsmath}
\newcommand{\BigO}[1]{\ensuremath{\operatorname{O}\bigl(#1\bigr)}}

\begin{document}
\title{Homework 4}
\author{Robbie McKinstry, Jack McQuown, Cyrus Ramavarapu}
\renewcommand{\today}{9 September 2016}
\renewcommand{\baselinestretch}{1.5}

\maketitle

\section*{Greedy Problems}
\subsection*{Problem 12:}
This greedy algorithm can be shown to be correct by an exchange argument.\\\\
Let $Alg$ be the process by which the greedy algorithm operates.  Assume that
there is some input $I$ such that $Alg(I)$ is incorrect. Let $Opt(I)$ be an
optimal solution that agrees with the most number of steps with $Alg(I)$.

\subsection*{Problem 18:}
\subsubsection*{A:}
\subsubsection*{B:}
\subsubsection*{C:}


\section*{Dynamic Programming}
\subsection*{Problem 1:}
\subsubsection*{A:}
\subsubsection*{B:}
To show that only \BigO{n^2} operations are needed if every duplicate
$T(i)$ is calculated only once, begin by expanding the sum in the 
recurrence.
\[
T(n) = \sum_{i=1}^{n-1}T(i)T(i-1)
\]
\[ 
T(n) = T(1)T(0) + T(2)T(1) + T(3)T(2)\dots
+ T(n-2)T(n-3) + T(n-1)T(n-2)
\]
Since every $T(i)$ will only be calculated once, following sequence
can be observed by counting the number of operations needed to determine
each $T(i)$.

\begin{center}
    \begin{tabular}{c| c c c c c}
    T (i) & T (2) & T (3) & T (4) & T (5) & T (6) \\ \hline  
    Ops & 1 & 3 & 5 & 7 & 9 \\
    \end{tabular}
\end{center}
It can be shown that the $T(i+1)$ element of the sum requires two additional
operations to calculate: \textit{a multiplication and an addition}.  Hence,
this sequence will continue.  It can be proven inductively that a closed
form expression for the sum of operations required is $n^2$.  Therefore,
in this case \BigO{n^2} operations are required. 

\subsubsection*{C:}
A \BigO{n} algorithm can be derived from the original recurrence
relationship by first eliminating the summation by calculating
$T(n+1)$ in the following manner.
\[
T(n+1) = \sum_{i=1}^{n}T(i)T(i-1)
\]
\[
T(n) = \sum_{i=1}^{n-1}T(i)T(i-1)
\]
\[
T(n+1) - T(n) = \sum_{i=1}^{n}T(i)T(i-1) - \sum_{i=1}^{n-1}T(i)T(i-1)  
\]
$T(n+1)$ and $T(n)$ overlap for all values $i:1\leq i\leq n-1$, therefore
subtracting the two sums leaves only the the final in the sum for $T(n+1)$.
\[
T(n+1) - T(n) = T(n)T(n-1)
\] 
The values for n can be shifted by setting $n = m-1$.
\[
T(m) - T(m-1) = T(m-1)T(m-2)
\]
However, the label $m$ is without meaning, so label $m=n$.
\[
T(n) - T(n-1) = T(n-1)T(n-2)
\]
Equivalently,
\[
T(n) = T(n-1)[1+T(n-2)]
\]
This expression is easily expressed as a single \BigO{n} loop.\\
\begin{algorithm}[H]
$Array:\ T$\\
$T[0] = 2$\\
$T[1] = 2$\\
\For{$i\leftarrow 2$ to $n$}
{$T[i] = T[i-1]*(1+T[i-2])$}
$Output:\ T[n]$

\end{algorithm}


\end{document}
