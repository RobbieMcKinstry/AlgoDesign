\documentclass[12pt]{article}

\usepackage[]{algorithm2e}
\usepackage{amsmath}
\newcommand{\BigO}[1]{\ensuremath{\operatorname{O}\bigl(#1\bigr)}}

\begin{document}
\title{Homework 4}
\author{Robbie McKinstry, Jack McQuown, Cyrus Ramavarapu}
\renewcommand{\today}{9 September 2016}
\renewcommand{\baselinestretch}{1.5}

\maketitle

\section*{Greedy Problems}
\subsection*{Problem 12:}


\subsection*{Problem 18:}


\section*{Dynamic Programming}
\subsection*{Problem 1:}
\subsubsection*{A:}
\subsubsection*{B:}
\subsubsection*{C:}
A \BigO{n} algorithm can be derived from the original recurrence
relationship by first eliminating the summation by calculating
$T(n+1)$ in the following manner.
\[
T(n+1) = \sum_{i=1}^{n}T(i)T(i-1)
\]
\[
T(n) = \sum_{i=1}^{n-1}T(i)T(i-1)
\]
\[
T(n+1) - T(n) = \sum_{i=1}^{n}T(i)T(i-1) - \sum_{i=1}^{n-1}T(i)T(i-1)  
\]
$T(n+1)$ and $T(n)$ overlap for all values $i:1\leq i\leq n-1$, therefore
subtracting the two sums leaves only the the final in the sum for $T(n+1)$.
\[
T(n+1) - T(n) = T(n)T(n-1)
\] 
The values for n can be shifted by setting $n = m-1$.
\[
T(m) - T(m-1) = T(m-1)T(m-2)
\]
However, the label $m$ is without meaning, so label $m=n$.
\[
T(n) - T(n-1) = T(n-1)T(n-2)
\]
Equivalently,
\[
T(n) = T(n-1)[1+T(n-2)]
\]
This expression is easily expressed in code.\\
\begin{algorithm}[H]
$Array:\ T$\\
$T[0] = 2$\\
$T[1] = 2$\\
\For{$i\leftarrow 2$ to $n$}{
$T[i] = T[i-1]*(1+T[i-2])$
}
$Output:\ T[n]$

\end{algorithm}


\end{document}
