\documentclass[12pt]{article}
\usepackage{amsmath, amssymb}
\usepackage{mathtools}
\usepackage{proof}
\usepackage{listings}
\usepackage{mathtools}
\usepackage{stmaryrd}
\usepackage{graphicx}

\renewcommand{\baselinestretch}{1.5}

\begin{document}
\vskip -3cm
\vskip 0.8cm

{\bf \centerline{Homework 2}}
\vskip 0.4cm

{\bf \centerline{Robbie McKinstry, Jack McQuown, and Cyrus Ramavarapu}}
\vskip .8cm

{\bf Question 4 Part A}

Let $S$ be the set of all gas stations. 
Let $G(A)$ be the output of the greedy algorithm on input $A$.
Assume $G_{A}(A)$ is not optimal for proof by contradiction.

Let $Opt$ be an optimal solution.

Consider the case where $Opt$ and $G(A)$ do not stop at the same gas stations. Since $\forall p \in S . G(A)$ stops at $p$., then $\exists p \in S . Opt$ does not stop at $p$.

Let $S$ be the gas station before $p$.

Let $Opt' = Opt$ such that $Opt$ stops at $S$ and fills the distance necessary to reach $p$, just as $G(A)$ does, and subtract that amount from the amount filled up at $p$.

$Opt'$ is at least as optimal as $Opt$ since the total amount of time spent filling is the same. It is obvious that $Opt'$ is more like $G(A)$ than $Opt$, which violates our assumption. $\bot$.

Thus, $G(A)$ is an optimal solution.

\vskip .6cm
{\bf Question 4 Part B}

This algorithm does not work. Consider the following counterexample:

Let $ F = 1 \text{liter} / 95 \text{kilo}$.
Let $C = 1 \text{liter}$.
Let $r = 1 \text{liter} / 95 \text{minutes}$.
Let the distance between $A$ and $B$ be $100 \text{kilos}$.
Let there be a single gas station 95 miles from A and 5 miles from B (that is, at the 95th mile marker).

This greedy algorithm specified will stop at the gas station and fill the tank 1 liter. Filling the tank 1 liter takes 95 minutes.
An optimal solution is to stop at the gas station and fill the tank $1/19$ liters, which will only take 5 minutes.

\vskip .6cm
{\bf Question 5 Part A}

We believe we have a counterexample.

Consider the set $A = \{ 0.25, 1, 1.25, 1.75, 2, 2.75 \}$

The greedy algorithm can cover four points by selecting the interval $[1, 2]$. Then, the greedy algorithm must cover the remaining two points individually, selecting an interval that covers $0.25$ and another interval that covers $2.25$, resulting in a total of $3$ intervals used.

However, an optimal solution would cover $A$ in two intervals, selecting $[0.25, 1.25]$ and $[1.75, 2.25]$ . Thus, this greedy algorithm is not a solution.

\vskip .6cm
{\bf Question 5 Part B}

Let $G(A)$ be the output of this greedy algorithm on input $A$. For proof by contradiction, assume $G$ is not optimal. Let $Opt(A)$ be an optimal output on input A that is closest in selection order to $G(A)$.

Let $i$ be the index of the first disagreement between $G(A)$ and $Opt(A)$, let $G(A)_{i}$ be the interval $[x_{i}, x_{i} + 1]$ selected by $G$, and $Opt(A)_{i}$ be the interval $[x'_{i}, y'_{i}]$ selected by $Opt$.

Since the intervals are not identical, either $x_{i} < x'_{i}$, or $x_{i} > x'_{i}$. 

Consider the case where $x_{i} < x'_{i}$:

Since $x_{i}$ is the leftmost point, and $x'_{i}$, $Opt$ has yet to cover $x_{i}$. Thus, there is an index $j : j > i$ where $Opt(A)_{j}$ covers $x'_{i}$.

Let $Opt(A)' = Opt(A) - Opt(A)_{j} + G(A)_{i}$. Now, it is obvious that $Opt(A)'$ has the same cardinality as $Opt(A)$. $Opt(A)'$ is still an optimal solution, because it has the optimal cardinality, and because $Opt(A)'$ doesn't cover any points left uncovered by $G(A)_{i}$; there are no points less than $x_{i}$ because if there were, they would have been selected by greedy algorithm, and there are no points greater than $x_{i}$ covered by $Opt(A)_{j}$ not covered by $G(A)_{i}$ because $G(A)_{i}$ covers all points within the range of 1 full interval, since it covers $[x_{i}, x_{i} + 1]$ by definition. Thus, $Opt(A)'$ is an optimal solution, which violates the premise that $Opt(A)$ is the optimal solution closest to $G$. $\bot$.

Now, consider the case where $x_{i} > x'_{i}$:

Let $Opt(A)' = Opt(A) - Opt(A)_{i} + G(A)_{i}$. It is obvious that $Opt(A)'$ has the same cardinality as $Opt(A)$. $Opt(A)'$ does not leave any points less than $x_{i}$ uncovered because by the definition of the greedy algorithm $x_{i}$ is the leftmost point. $Opt(A)'$ does not leave any points greater than $x_{i}$ uncovered because any point covered by the interval $[x_{i}, y'_{i}]$ will be covered by $[x_{i}, x_{i} + 1]$ since $y'_{i}$ is necessarily less than $x_{i} + 1$. Thus, $Opt(A)'$ is an optimal solution, which violates the premise that $Opt(A)$ is the optimal solution closest to $G$. $\bot$.

Since both cases result in contradiction, by disjunctive elimination the assumption introduces contradiction, and thus the assumption is proven false and $G(A)$ is necessarily a solution.

\end{document}
