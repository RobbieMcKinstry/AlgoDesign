\documentclass[12pt]{article}
\usepackage{amsmath, amssymb}
\usepackage{mathtools}
\usepackage{proof}
\usepackage{listings}
\usepackage{mathtools}
\usepackage{stmaryrd}
\usepackage{graphicx}

\renewcommand{\baselinestretch}{1.5}

\begin{document}
\vskip -3cm
\vskip 0.8cm

{\bf \centerline{Homework 2}}
\vskip 0.4cm

{\bf \centerline{Robbie McKinstry, Jack McQuown, and Cyrus Ramavarapu}}
\vskip .8cm

{\bf Question 4 Part A}

Let $S$ be the set of all gas stations. 
Let $G(A)$ be the output of the greedy algorithm on input $A$.
Assume $G_{A}(A)$ is not optimal for proof by contradiction.

Let $Opt$ be an optimal solution.

There are two possible scenarios: Either $G(A)$ and $Opt$ select the same gas stations to stop at or they do not. First, we address the former before tackling the latter.

Assume that $G(A)$ and $Opt$ select all of the same gas stations.
Then, for an arbitrary gas station $p$, let the amount of time $G(A)$ spends filling up be $x$.
Now, since $Opt$ must also stop at that gas station, and $Opt \neq G(A)$, $Opt$ must spend more time filling up than $G(A)$, since $G(A)$ fills up the minimum at every stop. Thus, $Opt$ stops for some $y : x \leq y$.  Since $Opt$ and $G(A)$ stop at all the same gas stations, total time $t$ is
\[
t_{Opt} = \sum_{p \in S} y_{p} 
\]
for $Opt$, and 

\[
t_{G(A)} = \sum_{p \in S} x_{p} 
\] for $G(A)$. By the properties of addition, $ t_{G(A)} \leq t_{Opt}$, and thus is an optimal solution. $\bot$.
Since we have resolved a contradiction, our assumption must be false. Thus, $G(A)$ is optimal and a solution, in the case where $G(A)$ and $Opt$ stop at all of the same gas stations.

Now, we address the case where $Opt$ and $G(A)$ do not stop at the same gas stations. Since $\forall p \in S . G(A)$ stops at $p$., then $\exists p \in S . Opt$ does not stop at $p$.

Let $S$ be the gas station before $p$.

Let $Opt' = Opt$ such that $Opt$ stops at $S$ and fills the distance necessary to reach $p$, just as $G(A)$ does, and subtract that amount from the amount filled up at $p$.

$Opt'$ is at least as optimal as $Opt$ since the total amount of time spent filling is the same. It is obvious that $Opt'$ is more like $G(A)$ than $Opt$, which violates our assumption. $\bot$.

Thus, $G(A)$ is an optimal solution.

\vskip .6cm
{\bf Question 4 Part B}

This algorithm does not work. Consider the following counterexample:

Let $ F = 1 \text{liter} / 95 \text{kilo}$.
Let $C = 1 \text{liter}$.
Let $r = 1 \text{liter} / 95 \text{minutes}$.
Let the distance between $A$ and $B$ be $100 \text{kilos}$.
Let there be a single gas station 95 miles from A and 5 miles from B (that is, at the 95th mile marker).

The greedy algorithm specified will stop at the gas station and fill the tank 1 liter. Filling the tank 1 liter takes 95 minutes.
An optimal solution is to stop at the gas station and fill the tank $1/19$ liters, which will only take 5 minutes.

\vskip .6cm
{\bf Question 5 Part A}


\vskip .6cm
{\bf Question 5 Part B}

\end{document}
