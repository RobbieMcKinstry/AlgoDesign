\documentclass[12pt]{article}

\usepackage{amsmath}

\begin{document}
\title{Homework 3}
\author{Robbie Mckinstry, Jack, Cyrus}
\renewcommand{\today}{4 September 2016}
\renewcommand{\baselinestretch}{1.5}

\maketitle

\section*{Problem 9:}
\subsection*{A}

The algorithm whereby the skier and ski whose height difference
is minimized gets assigned first and then the process is repeated
is incorrect.


Consider the following counter example:

\[
\text{Skiers } = \{1,2,3\}
\]
\[
\text{Skis } = \{2.5, 2.6, 3.6\}
\]

This greedy algorithm will produce the following pairings:

\[
\text{(Skiers, Skis) } = \{(1,3.6),(2,2.5),(3,2.6)\}
\]

This has an average height difference of $3.5/3$.

An optimal pairing would be:
\[
\text{(Skiers, Skis) } = \{(1,2.5),(2,2.6),(3,3.6)\}
\]

This has an average height difference of $2.7/3$.

\subsection*{B}
This greedy algorithm can be shown to be correct using an
exchange argument.

Let $Alg$ be the process by which the greedy algorithm operates.
Assume there exists some input $I$ such that $Alg(I)$ is incorrect.

Let $Opt(I)$ be the optimal output for input $I$ that agrees with
$Alg(I)$ for the greatest number of steps.

Since $Opt(I)$ cannot equal $Alg(I)$, there must be an earliest
step of disagreement.  Label this step $i$.\\\\
At step $i$, let $Alg(I)$ select $(x_a, y_a)$ and let $Opt(I)$ select
$(x_i,y_i)$.\\\\
Since for all steps $t < i$, $Opt_t(I) = Alg_t(I)$, the values $x_a$ and
$y_a$ must be used at some step after $i$ in $Opt(I)$; however, $x_a$ and
$y_a$ need not be used together in the same step.\\\\
Additionally, since $Alg(I)$ always picks the two smallest available
elements and pairs them together, the pair $x_a$ and $y_a$ are the smallest
possible values available at step $i$.  Therefore, in $Opt(I)$, $x_a$ and
$y_a$ are respectively smaller than any other $x$ and $y$ used by $Opt(I)$
after step $i$.\\\\
Let the step beyond $i$ that $Opt(I)$ uses
$x_a$ and $y_a$ be respectively $u$ and $v$.  Without loss of generality,
let $u < v$.  Define $Opt'(I)$ as $Opt(I)$ execept for all steps $k: i \leq 
k \leq v$ sort the $x$ and $y$ values in ascending order.  Since $x_a$ and $y_a$
are the minimum values in the range $k: i \leq k \leq v$, they will be placed
at step $i$.  Hence, $Opt'(I)$ agrees with $Alg(I)$ for at least one more
step than $Opt(I)$.\\\\
To show that $Opt'(I)$ is at least as optimal as $Opt(I)$\dots \textit{Working on this 
now peeps.}    

\end{document}
