\documentclass[12pt]{article}

\usepackage{amsmath}

\begin{document}
\title{Homework 3}
\author{Robbie McKinstry, Jack McQuown, Cyrus Ramavarapu}
\renewcommand{\today}{4 September 2016}
\renewcommand{\baselinestretch}{1.5}

\maketitle

\section*{Problem 9:}
\subsection*{A}

The algorithm whereby the skier and ski whose height difference
is minimized gets assigned first and then the process is repeated
is incorrect.


Consider the following counter example:

\[
\text{Skiers } = \{1,2,3\}
\]
\[
\text{Skis } = \{2.5, 2.6, 3.6\}
\]

This greedy algorithm will produce the following pairings:

\[
\text{(Skiers, Skis) } = \{(1,3.6),(2,2.5),(3,2.6)\}
\]

This has an average height difference of $3.5/3$.

An optimal pairing would be:
\[
\text{(Skiers, Skis) } = \{(1,2.5),(2,2.6),(3,3.6)\}
\]

This has an average height difference of $2.7/3$.

\subsection*{B}
This greedy algorithm can be shown to be correct using an
exchange argument.\\\\
Let $Alg$ be the process by which the greedy algorithm operates.
Assume there exists some input $I$ such that $Alg(I)$ is incorrect.
Let $Opt(I)$ be the optimal output for input $I$ that agrees with
$Alg(I)$ for the greatest number of steps.\\\\
Since $Opt(I)$ cannot equal $Alg(I)$, there must be an earliest
step of disagreement.  Label this step $i$.\\\\
At step $i$, let $Alg(I)$ select $(x_a, y_a)$ and let $Opt(I)$ select
$(x_i,y_i)$.\\\\
Since for all steps $t < i$, $Opt_t(I) = Alg_t(I)$, the values $x_a$ and
$y_a$ must be used at some step after $i$ in $Opt(I)$; however, $x_a$ and
$y_a$ need not be used together in the same step.\\\\
Additionally, since $Alg(I)$ always picks the two smallest available
elements and pairs them together, the pair $x_a$ and $y_a$ are the smallest
possible values available at step $i$.  Therefore, in $Opt(I)$, $x_a$ and
$y_a$ are respectively smaller than any other $x$ and $y$ used by $Opt(I)$
after step $i$.\\\\
Let the step beyond $i$ that $Opt(I)$ uses
$x_a$ and $y_a$ be respectively $u$ and $v$.  Without loss of generality,
let $u < v$.  Define $Opt'(I)$ as $Opt(I)$ execept for all steps $k: i \leq 
k \leq v$ sort the $x$ and $y$ values in ascending order.  This sort is done
independently on the $x$ and $y$ values.  For example, if $x_i > x_j$ and
$y_i < y_j$, $Sort((x_i,y_i),(x_j,y_j))\to((x_j,y_i),(x_i,y_j))$.\\\\
Since $x_a$ and $y_a$
are the minimum values in the range $k: i \leq k \leq v$, they will be placed
at step $i$.  Hence, $Opt'(I)$ agrees with $Alg(I)$ for at least one more
step than $Opt(I)$.  $Opt'(I)$ is also a feasible solution since every element
is still used.\\\\
To show that $Opt'(I)$ is at least as optimal as $Opt(I)$ upon sorting requires
showing that $|x_i - y_i| + |x_j - y_j|$ does not increase after the sort.  This
follows from the following cases which fix $y_i < y_j$.  The cases where
$y_i$ is not less than $y_j$ are analogous since a conflict only arises when both $x$ and $y$
are sorted.\\\\\\\\  
\textit{Case 1:} $x_i < y_i$\\
In this case $x_j < x_i < y_i < y_j$.  Therefore, $(x_i - y_i) < 0$
and $(x_j - y_j) < 0$.  By the defintion of the absolute value, the following holds
true:\\
$|x_i - y_i| + |x_j - y_j| = -(x_i - y_i) - (x_j - y_j) = -x_i + y_i - x_j + y_j$\\
This can be rearranged and the absolute values can be replaced, giving: $|x_i - y_j| + |x_j - y_i|$.\\
This is equivalent to the sum if the values were sorted, thereby showing that the value
of the sum does not change under sorting.\\\\
\textit{Case 2:} $x_i = y_i$\\
Since $x_i = y_i$, $(x_i - y_i) = 0$. The sum can therefore be expanded as
$-(x_i - y_i) - (x_j - y_j)$ since $x_j < x_i$.  Rearranging these values
give and adding the absolute values gives $|x_i - y_j| + |x_j - y_i|$.\\\\ 
\textit{Case 3:} $x_j < y_i < x_i < y_j$\\
Based on the inequality, the defintion of the absolute value function can be used to show
$|x_i - y_i| + |x_j - y_j| = x_i - y_i - (x_j - y_j) = x_i - y_i - x_j + y_j$.  Since 
$y_i < y_j$, $x_i - y_i - x_j + y_j > x_i + y_i - x_j - y_j$ because the positive quantity
$(y_j - y_i)$ has been replaced with the negative quantity $(y_i - y_j)$.  This lower bound
can be rewritten as $|x_i - y_j| + |x_j - y_i|$, thereby showing that the sorting of the
values actually reduces the sum.\\\\
Cases such as $y_i < x_j < x_i < y_j$ and $y_i < y_j < x_j < x_i$ have similar explanations
showing the sum after a sort being equal to or less than the unsorted sum.\\\\
Since sorting will produce at least as optimal a result as $Opt(I)$, $Opt'(I)$ is at least
as optimal as $Opt'(I)$, but agrees with more steps than $Opt(I)$ which is a contradiction,
proving that there is no input $I$ for which the greedy algorithm $Alg$ is incorrect.     


\end{document}
