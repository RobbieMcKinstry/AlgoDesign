\documentclass[12pt]{article}

\usepackage{amsmath}

\begin{document}
\title{Greedy Problem 9}
\renewcommand{\today}{4 September 2016}
\renewcommand{\baselinestretch}{1.5}

\maketitle

\section{Problem 9:}
\subsection{A}

The algorithm whereby the skier and ski whose height difference
is minimized gets assigned first and then the process is repeated
is incorrect.


Consider the following counter example:

\[
\text{Skiers } = \{1,2,3\}
\]
\[
\text{Skis } = \{2.5, 2.6, 3.6\}
\]

This greedy algorithm will produce the following pairings:

\[
\text{(Skiers, Skis) } = \{(1,3.6),(2,2.5),(3,2.6)\}
\]

This has an average height difference of $3.5/3$.

An optimal pairing would be:
\[
\text{(Skiers, Skis) } = \{(1,2.5),(2,2.6),(3,3.6)\}
\]

This has an average height difference of $2.7/3$.

\subsection{A}
This greedy algorithm can be shown to be correct using an
exchange argument.

Let $Alg$ be the process by which the greedy algorithm operates.
Assume there exists some input $I$ such that $Alg(I)$ is incorrect.

Let $Opt(I)$ be the optimal output for input $I$ that agrees with
$Alg(I)$ for the greatest number of steps.

Since $Opt(I)$ cannot equal $Alg(I)$, there must be an earliest
step of disagreement.  Label this step $i$.

\end{document}
