\documentclass[12pt]{article}
\usepackage{amsmath, amssymb}
\usepackage{mathtools}
\usepackage{proof}
\usepackage{listings}
\usepackage{mathtools}
\usepackage{stmaryrd}

\renewcommand{\baselinestretch}{1.5}

\begin{document}
\vskip -3cm
\vskip 0.8cm

{\bf \centerline{Homework 1}}
\vskip 0.4cm

{\bf \centerline{Robbie McKinstry, Jack McQuown, and Cyrus Ramavarapu}}
\vskip .8cm

{\bf Question 1}

Proof by contradiction:

Assume that there is an input J on which the greedy algorithm is incorrect.

Let $Alg(J)$ = the output on input J.
Let $Opt(J)$ be the optimal solution on input J.

Let $i$ be index of the output sequence generated by the picking rule for each algorithm, and let $Alg_{i}$ be the $i$ element in $Alg(J)$ output sequence, and $Opt_{i}$ be the $i$th element in the $Opt(J)$ output sequence.

Let $k$ be the first index in the output sequence where $Alg_{k} \neq Opt_{k}$.

Since $Alg$ and $Opt$ picked different elements, $Alg_{k}$ overlaps with fewer elements than $Opt_{k}$. This means that for the next selection $k+1$, Alg has more elements in the pool of selectable elements than $Opt$. Since there are at least as many elements available in the set of selectable elements, $Alg$ can always select another element if $Opt$ can select another element. This proves that $Alg$ will always have the same cardinality as $Opt$ when selecting element $k+1$. Since $Alg$ and $Opt$ have the same cardinality, $Alg$ is correct. $\bot$. Thus, our assumption is wrong, and $Alg$ must be a correct solution to the problem.

\vskip .6cm
{\bf Question 2 Part A}

We believe that the algorithm is a solution.

We believe this because, as noted, the algorithm cannot have a number of colors less than $s$, which is fairly obvious. It also cannot have a number of colors more than $s$ because each of the colors are necessarily disjoint. The colors are disjoint because if they were not, that would violate the picking criteria which eliminates intervals that are not disjoint. Since they are disjoint, recoloring an interval would result in a collision.

\vskip .6cm
{\bf Question 2 Part B}

We believe that the algorithm is a solution.

We believe this because, as noted, the algorithm cannot have a number of colors less than $s$, which is fairly obvious; otherwise, there would necessarily be a conflict. It also cannot have a number of colors more than $s$ because before coloring an interval with a new color, we look to see whether or not it could be colored with a pre-existing color. The only time where it would conflict is if all other rooms had an interval colored already. This is, by definition, $s$.

\vskip .6cm
{\bf Question 3 Part A}

The shillings greedy algorithm is not correct. As a counter example, consider $40$ shillings. The greedy algorithm would occasional would select $25$, then $10$, then $5$ to make $40$. However, an optimal solution would be to take $20$ twice.

{\bf Question 3 Part B}

Proof by contradiction:

Assume that there is an input $I$ on which the greedy algorithm is incorrect. Let $Opt(I)$ be a solution. Let $G(I)$ be the output of the greedy algorithm.

Let $i$ be the first coin on which $Opt$ and $G$ disagree. Since $G$ always takes the highest valued coin, that means that $G_{i} > Opt_{i}$. Thus, $Opt$ must select from the remaining coins a selection which sums to $G_{i}$. Then, $Opt$ has to take at least $2$ of a single coin from the remaining coins, because $1$ of each remaining coin is necessarily less than the value of $G_{i}$ \textit{i.e.} $2^{n} > 2^{n-1} + 2^{n-2} ... + 2^{0}$. 

If an algorithm were to select a single coin twice, then it would be able to select that same amount with fewer coins by taking the least-valued coin greater than the selected coin once \textit{e.g.} $2 * 2^{k} = 2^{k+1}$. Thus, any algorithm that takes two of a single coin is not a solution, since there exists an algorithm that could take one fewer coin by exchanging the duplicates for a single higher valued coin. Since $Opt$ takes $2$ coins, it is not a solution. $\bot$.

Thus, our assumption is false, and the Greedy algorithm is correct.

Aside:

Since, there is a bijection

\end{document}
