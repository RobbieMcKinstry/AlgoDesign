\documentclass[12pt]{article}

\usepackage{tikz}
\usetikzlibrary{trees}
\usepackage{algpseudocode}
\usepackage{amsmath}
\newcommand{\BigO}[1]{\ensuremath{\operatorname{O}\bigl(#1\bigr)}}

\begin{document}
\title{Homework 5}
\author{Robbie McKinstry, Jack McQuown, Cyrus Ramavarapu}
\renewcommand{\today}{12 September 2016}
\renewcommand{\baselinestretch}{1.5}

\maketitle

\section*{Greedy Problems}
\subsection*{Problem 7:}

\subsection*{Problem 17:}

\section*{Dynamic Programming}
\subsection*{Problem 2:}
The longest common subsequence between the three can be
initially defined recursively by considering the last 
letter in common between the three strings and then 
seeing if this letter is in the longest common subsequence
of one letter shorter.
\[
S_A = A_1, A_2, A_3,\dots,A_{i-1},A_i
\]
\[
S_B = B_1, B_2, B_3,\dots,B_{j-1},B_j
\]
\[
S_C = C_1, C_2, C_3,\dots,C_{k-1},C_k
\]
%Insert sentence here 
This analysis leads to the following recursive algorithm:\\\\
\begin{algorithmic}[H]

\Function{LCS}{$int$ $i$,$j$,$k$}:
\If{$i\equiv j\equiv k\equiv0$}
    \State $return\ 0$
\EndIf\\
\If{$A_i\equiv B_j\equiv C_k$}
    \State$LCS(i-1,j-1,k-1) + A_i$
\Else 
    \If{$(A_i\equiv B_j)\neq C_k$}
        \State$max(LCS(i-1,\ j-1,\ k),\ LCS(i,\ j,\ k-1))$
    \Else
        \If{$(A_i\equiv C_k)\neq B_j$}
            \State $max(LCS(i-1,\ j,\ k-1),\ LCS(i, j-1, k))$
        \Else
            \If{$A_i\neq (B_j\equiv C_k)$}
                \State$max(LCS(i,\ j-1,\ k-1),\ LCS(i-1,\ j,\ k))$
            \Else
                % A_i\neq B_j\neq C_k)
                \State $max($\[LCS(i,\ j-1,\ k-1),\]\[LCS(i-1,\ j,\ k-1),\]\[LCS(i-1,\ j-1,\ k))\]
            \EndIf
       \EndIf
   \EndIf
\EndIf  
\EndFunction\\\\
Although this algorithm produces the longest common subsequence for
three given strings, it runs in exponential time due to the numerous
recursive calls that operate on a problem of only $1$ letter smaller.\\\\
By moving to an array based solution and changing the recursive calls 
to array look-ups, a polynomial runtime algorithm can be developed.  
%j -> i -> k order and lowest to highest indices
\end{algorithmic}
\end{document}
