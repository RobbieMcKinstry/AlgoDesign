\documentclass[12pt]{article}

\usepackage{tikz}
\usetikzlibrary{trees}
\usepackage[]{algorithm2e}
\usepackage{amsmath}
\newcommand{\BigO}[1]{\ensuremath{\operatorname{O}\bigl(#1\bigr)}}

\begin{document}
\title{Homework 5}
\author{Robbie McKinstry, Jack McQuown, Cyrus Ramavarapu}
\renewcommand{\today}{12 September 2016}
\renewcommand{\baselinestretch}{1.5}

\maketitle

\section*{Greedy Problems}
\subsection*{Problem 7:}
A greedy algorithm that will minimize evictions is to 
select the page that will be accessed furthest in the future
from the current time.\\\\
For example, assuming $k=4$ let the current state of the system be the
following:
\begin{center}
    \begin{tabular}{c|c|c|c|c|c|c|c}
    Time & 1 & 2 & 3 & 4 & 5 & 6 & 7 \\
    Input & A & C & D & A & B & A & A \\   
    \end{tabular}
    Fast Memory ($k=4$) 
    \begin{tabular}{c|c|c|c}
    E & D & F & B 
    \end{tabular}
\end{center}
In this situation, the algorithm will evict page B, since
it is not used until time 5.  Any other choice will result
in an earlier eviction. 
\subsubsection*{Proof by Exchange:}
Let $Alg$ be the process by which the above algorithm operates.
Assume there exists an input $I$ such that $Alg(I)$ is 
incorrect.  Let $Opt(I)$ be the optimal result for $I$
that agrees with the greatest number of steps with $Alg(I)$.
$Alg(I)$ and $Opt(I)$ must have a first point of disagreement.
Label this time $t_i$.\\\\
At time $i$, $Alg(I)$ must have evicted some page $u$ and 
$Opt(I)$ has some page $v$ evicted.  Since $Alg(I)$ always
picks the page that will be used furthest in the future,
the next time page $u$ will be used must be further into
the future than the next time page $v$ is used.  Let these
times be respectively $t_u$ and $t_v$.\\\\
Since $Alg(I)$ and $Opt(I)$ both agreed upto time $t_i$,
pages $u$ and $v$ must be in the fast memory of $Opt(I)$
at $t_i$.  Therefore, define $Opt'(I)$ as $Opt(I)$ except
at time $t_i$ evict page $u$ instead of $v$.\\\\
To show that $Opt'(I)$ is at least as optimal as $Opt(I)$     
it needs to be recognized that since $t_u > t_v$, page $v$
will have to be brought back into memory at least as many
times as page $u$ before
page $u$ is brought back into memory.\\\\
Additionally, due to the fact that the pages in $Opt'(I)$
and $Opt(I)$ are the same except for the page needed at
$t_u$ is repalced with that at $t_v$, $Opt'(I)$ will not
evict a page that $Opt(I)$ will not evict until at least
after time $t_u$.  This means that there will be no
more evictions in $Opt'(I)$ than in $Opt(I)$.   
Since $Opt'(I)$ is at least as optimal as $Opt(I)$ and 
agrees with $Alg(I)$ for $1$ more step a contradiction has
be reached.  Therefore, there is not input for which $Alg$
is incorrect.   
\subsection*{Problem 17:}
A greedy algorithm that will determine if there is enough
information available to fairly partition the goods is as 
follows:\\\\
Given a set of goods $\mathcal{G}={G_1,\ G_2,\dots,G_n}$,
and two orderings on this set, $\mathcal{H}=G_a > G_b >
\cdots > G_n$ and $\mathcal{W}=G_i > G_k > \cdots > G_m$,
partition the goods by initially giving $\mathcal{W}$ and 
$\mathcal{H}$ their maximal elements.
Remove the element given to the opposing list from each list.
If this is not possible
because the maximal elements for both $\mathcal{W}$ and
$\mathcal{H}$ are the same, conclude \textit{that there is not
enough information to guarantee a fair partitioning}.\\\\  
While there are elements in $\mathcal{G}$, continue
giving $\mathcal{W}$ and $\mathcal{H}$ their second
most desired element.  Ties are broken arbitrarily.  For 
example, if at a given step both $\mathcal{W}$ and $\mathcal{H}$
want $G_c$, arbitrarily give $G_c$ to $\mathcal{W}$ and
cross it off in the ordering for $\mathcal{H}$ and give
$\mathcal{H}$ the next element in the ordering.\\\\
After all elements have been assigned,first evenly parition  
$\mathcal{H}$ and $\mathcal{W}$ into two equal portions.
If this is not possible because $|\mathcal{G}|$ is odd,
conclude that \textit{there is not enough information available
to fairly partition the goods}.  If the upper half of  
$\mathcal{H}$ and $\mathcal{W}$ is full, conclude that 
\textit{there is enough information to fairly partition}.\\\\
If the upper half of $\mathcal{H}$ or $\mathcal{W}$ are
not full, begin iterating through both orderings and 
check to see that items given alternate with items not given.
Repeat this process for alternation in groups of $2,\ 3,\dots
,\ |\mathcal{G}|/2$.  Every assigned element must belong
to at least $1$ of the alternating patterns.
If an assignment was reached such that elements from $\mathcal{G}$
can be given to $\mathcal{H}$ and $\mathcal{W}$ such that both
pass at least one test, conclude that \textit{there is enough
information to fairly partition the goods}.  Otherwise, conclude
that \textit{there is not enough information available}.
\subsubsection*{Proof by Exchange:}
This algorithm can be shown to be correct by an exchange argument.
Let $Alg$ be the process by which the above algorithm operates.
Assume there exists some input $I$ such that $Alg(I)$ is incorrect.
Let $Opt(I)$ be the optimal output for input $I$ that aggrees with
$Alg(I)$ for the most number of steps.  Therefore, $Alg(I)$ and $Opt(I)$
must have a first point of disagreement.  Label this step $s_i$.\\\\
At $s_i$, $Alg(I)$ must have given a different pair of goods to
$\mathcal{H}$ and $\mathcal{W}$ than $Opt(I)$.  This could mean that
$Alg(I)$ respectively gave $\mathcal{H}$ and $\mathcal{W}$ $G_a,\ G_b$      
whereas $Opt(I)$ gave $G_c,\ G_d$.  However, since $Alg$ attempts
to assign the maximal element at each step to $\mathcal{H}$ and 
$\mathcal{W}$ and since $Alg(I)$ and $Opt(I)$ agree upto 
step $s_i$, the items $G_a,\ G_b$ must be assigned in $Opt(I)$
at a step, $s_j$, after $s_i$.  As a result, define $Opt'(I)$
as $Opt(I)$ except at $s_i$ assign $G_a$ and $G_b$ to the
same lists as $Alg(I)$ and move the items originally assigned
in $Opt(I)$ at $s_i$ to $s_j$.\\\\
To check that $Opt'(I)$ is at least as optimal as $Opt(I)$,
the case in which $Opt(I)$ determines that there is enough
information to partition the goods and the case in which it
finds the information insufficient need to be considered 
separately.  In the first case, the condition for enough
information requires the sequence of assigned goods and
unassigned goods follow an alternating pattern.  Since 
$Alg(I)$ and $Opt(I)$ agreed until step $s_i$ alternation
must have already started and to continue, $G_a,\ G_b$ and 
$G_c,\ G_d$ must have been in alternating positions.  Exchanging
these values therefore does not change the pattern, giving
the same result.\\\\    
In the case that $Opt(I)$ finds the ordering information
insufficient to fairly partition the goods, there will be
an element that does not belong to any alternating pattern 
of the elements assigned to $\mathcal{H}$
and $\mathcal{W}$.  Since the elements assigned by $Opt(I)$ at
$s_i$ are not the maximal elements available, swaping $G_a,\ G_b$
instead of $G_c,\ G_b$, will not change the patterning since
the elements are ordered.  As a result, $Opt'(I)$ will also
conclude that there is not enough information available.\\\\
Since in both cases $Opt'(I)$ is at least as optimal as $Opt(I)$
and agrees with $Alg(I)$ for one more step than $Opt(I)$, a 
contradiction has been reached.  Therefore there does not exist 
an input for which $Alg(I)$ is incorrect.   
\section*{Dynamic Programming}
\subsection*{Problem 2:}
The longest common subsequence between the three can be
initially defined recursively by considering the last 
letter in common between the three strings and then 
seeing if this letter is in the longest common subsequence
of one letter shorter.
\[
S_A = A_1, A_2, A_3,\dots,A_{i-1},A_i
\]
\[
S_B = B_1, B_2, B_3,\dots,B_{j-1},B_j
\]
\[
S_C = C_1, C_2, C_3,\dots,C_{k-1},C_k
\]
Reading right to left, the first letter common to each string will
be the last letter in the longest common subsequence.  As a result,
once this letter is found, problem can be redefined in terms of the 
shorter substrings formed by ignoring the first common point and 
all suceeding letters.\\\\
This analysis leads to the following recursive algorithm:\\\\
\begin{algorithm}[H]
\SetKw{Func}{Function:}
\SetKw{Inp}{Input:}
\Func{LCS\\}
\Inp{$int\ i$, $int\ j$, $int\ k$}\\
\If{$i\equiv j\equiv k\equiv0$}
{$return\ 0$}
\uIf{$A_i\equiv B_j\equiv C_k$}
{$LCS(i-1,j-1,k-1) + A_i$}
\uElseIf{$(A_i\equiv B_j)\neq C_k$}
{$\max(LCS(i-1,\ j-1,\ k),\ LCS(i,\ j,\ k-1))$}
\uElseIf{$(A_i\equiv C_k)\neq B_j$}
{$\max(LCS(i-1,\ j,\ k-1),\ LCS(i, j-1, k))$}
\uElseIf{$A_i\neq (B_j\equiv C_k)$}
{$\max(LCS(i,\ j-1,\ k-1),\ LCS(i-1,\ j,\ k))$}
\Else{$\max(LCS(i,\ j-1,\ k-1),LCS(i-1,\ j,\ k-1),LCS(i-1,\ j-1,\ k))$}
\end{algorithm}
Although this algorithm produces the longest common subsequence for
three given strings, it runs in exponential time due to the numerous
recursive calls that operate on a problem of only $1$ letter smaller.\\\\
By moving to an array based solution and changing the recursive calls 
to array look-ups, a polynomial runtime algorithm can be developed.\\\\
\begin{algorithm}[H]
\SetKw{Func}{Function:}
\SetKw{Inp}{Input:}
\SetKw{Init}{Initialization:}
\SetKw{Calc}{Array Calculations:}
\SetKw{Arr}{Array$[\ ][\ ][\ ]$}
\Func{Array LCS\\}
\Inp{$string\ A$, $String\ B$, $String\ C$}\\ 
\Init\\
\Arr{LCS}\\
\For{$i\leftarrow 0$ to $len(A)$}
{$LCS[i][0][0] = 0$}
\For{$j\leftarrow 0$ to $len(B)$}
{$LCS[0][j][0] = 0$}
\For{$k\leftarrow 0$ to $len(C)$}
{$LCS[0][0][k] = 0$}
\Calc\\
\For{$i\leftarrow 0$ to $len(A)$}
{\For{$j\leftarrow 0$ to $len(B)$}
{\For{$k\leftarrow 0$ to $len(C)$}
{\uIf{$A_i\equiv B_j\equiv C_k$}
{$LCS[i][j][k] = LCS[i-1][j-1][k-1] + 1$}
\uElseIf{$(A_i\equiv B_j)\neq C_k$}
{$LCS[i][j][k]=\max(LCS[i-1][j-1][k],\ LCS[i][j][k-1])$}
\uElseIf{$(A_i\equiv C_k)\neq B_j$}
{$\max(LCS[i][j][k]=LCS[i-1][j][k-1],\ LCS[i][j-1][k])$}
\uElseIf{$A_i\neq (B_j\equiv C_k)$}
{$LCS[i][j][k] = \max(LCS[i][j-1][k-1],\ LCS[i-1][j],[k])$}
\Else{$LCS[i][j][k] = \max(LCS[i][j-1][k-1]$,\\$LCS[i-1][j][k-1],\ LCS[i-1][j-1][k])$}
}
}
}
\end{algorithm}
By tracing backwards from $LCS[len(A)-1][len(B)-1][len(C)-1]$
following the path of the largest lengths, the \textit{String}
value of the longest common subsequence can be recovered.

\end{document}
