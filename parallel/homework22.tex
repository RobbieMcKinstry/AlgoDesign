\documentclass[12pt]{article}

\usepackage{tikz}
\usepackage[ruled, vlined]{algorithm2e}
\usepackage{pst-node,pst-plot}
\usepackage{amsmath}
\usepackage{float}
\usepackage{algorithm2e}

\newcommand{\BigO}[1]{\ensuremath{\operatorname{\mathcal{O}}\bigl(#1\bigr)}}

\begin{document}
\title{Homework 22}
\author{Robbie McKinstry, Jack McQuown, Cyrus Ramavarapu}
\renewcommand{\today}{24 October 2016}
\renewcommand{\baselinestretch}{1.5}
\maketitle

\section*{Problem 4: }
The algorithm that can be used in order to get parallel prefix to run in {O(log n)} time is called Scan. It is an established parallel 
pattern that is used when the main goal is to compute partial reductions of the entire collection, however in order to Scan to work 
correctly the operators used must be associative.\\\\
The basic idea of scan is to first compute the sums of every two elements in the collection, i.e. (1+2), (3+4), (5+6), {$\cdots$} The second step is to then sum these values with each other, so (1+2)+(3+4), (5+6)+(7+8), {$\cdots$} The third step is to then bring the first element down into its respective spot in the final collection, and continue the pattern of adding the previous summations
[(1+2)+(3+4)]+[(5+6)+(7+8)]. The fourth step is to then take [(1+2)+(3+4)]+(5+6). The final step is to then add [(1+2)]+3, [(3+4)]+5, and [(5+6)]+7.\\\\
This explanation is better illustrated with the table below\\
\begin{table}[]
\centering
\caption{Scan Table}
\label{my-label}
\begin{tabular}{c|c|c|c|c|c|c|c|c|}
                       & 1              & 2       & 3           & 4       & 5                       & 6     & 7                         & 8                                   \\\cline{2-9} 
 & {$\searrow$} &{$\downarrow$}&{$\searrow$}&{$\downarrow$}&{$\searrow$}&{$\downarrow$}&{$\searrow$}&\\\cline{2-9}
1 &                & (1+2)   &             & (3+4)   &                         & (5+6) &                           & (7+8)\\\cline{2-9}
&                &{$\searrow$}&             &{$\swarrow$}&&{$\searrow$}& &{$\swarrow$} \\\cline{2-9}
2&                &         & (1+2)+(3+4) &         &                         &       & (5+6)+(7+8)               &  \\\cline{2-9}
&&         &{$\downarrow$}&{$\rightarrow$}&{$\rightarrow$}&{$\rightarrow$}& {$\searrow$}&   \\\cline{2-9}
3& 1              &         &{$\downarrow$}  &         &                         &       &                           & $\sum\limits_{i=1}^{8} i$\\\cline{2-9}
& {$\downarrow$} &         &             & {$\searrow$} &                         &       &                           &        {$\downarrow$} \\\cline{2-9}
4&{$\downarrow$}&         &             &         &  &       &                           &   {$\downarrow$} \\\cline{2-9}
& {$\downarrow$}&         &             &         &                         &       &                           &{$\downarrow$}  \\\cline{2-9}
5& {$\downarrow$}& (1+2)+3 &             & (3+4)+5 &                         &       &  &      {$\downarrow$} \\\cline{2-9}
&                &         &             &         &                         &       &                           &                                    
\end{tabular}
\end{table}
\section*{Problem 6: }
\section*{Problem 7: }
\section*{Problem 8: }


\end{document}
