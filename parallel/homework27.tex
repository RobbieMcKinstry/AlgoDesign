\documentclass[12pt]{article}

\usepackage{tikz}
\usetikzlibrary{calc,shapes.multipart,chains,arrows}
\usepackage[ruled, vlined]{algorithm2e}
\usepackage{pst-node,pst-plot}
\usepackage{amsmath}
\usepackage{float}

\newcommand{\BigO}[1]{\ensuremath{\operatorname{\mathcal{O}}\bigl(#1\bigr)}}

\begin{document}
\title{Homework 27}
\author{Robbie McKinstry, Jack McQuown, Cyrus Ramavarapu}
\renewcommand{\today}{04 November 2016}
\renewcommand{\baselinestretch}{1.5}
\maketitle

\section*{Problem 21: }
On an EREW PRAM, the leave nodes of an arbitrary tree with $n$ nodes
can be numbered in order in \BigO{\log n} time using $n$ processors
if initially each of the $n$ processors is assigned to a unique node.\\\\
Initially, each processor will have to determine whether or not it is
assigned a leaf node.  This can be done in parallel in \BigO{1} time
by having each processor check if has any children.  If the processor
has children, it will hold the value $0$.  If the processor is 
assigned to a leaf node it will hold the value $1$.\\\\
After numbering all the internal nodes as $0$ and all the leaf nodes as
$1$, an \textit{Eulerian Tour} can be performed starting from the root
of the tree headed left.  Each edge will be marked with a $+0$ and the
result of the tour will be a linked list where each node will either
contribute $1$ to the sum, if it is a leaf, or a $0$ if it is an internal
node.  The parallel prefix problem can then be solved on this linked list,
and the resulting sequence of consecutive numbers will number the leaves
from left to right.\\\\
To demonstrate this algorithm, consider the following tree:\\\\  
\begin{center}
\begin{tikzpicture}[level distance=1.5cm,
    level 1/.style={sibling distance=3.40cm},
    level 2/.style={sibling distance=1.25cm},
    level 3/.style={sibling distance=1.15cm}]
    \node {$A$}
        child {node {$B$}
            child {node {$C$}}
            child {node {$D$}
                child {node {$E$}}
                child {node {$F$}}
            }
        }
        child {node {$G$}
        };
\end{tikzpicture}
\end{center} 
After the initial step that will number each node as either a $0$ or
$1$ depending if the node is a leaf or an internal node, the tree
will look as follows:\\\\ 
\begin{center}
\begin{tikzpicture}[level distance=1.5cm,
    level 1/.style={sibling distance=3.40cm},
    level 2/.style={sibling distance=1.25cm},
    level 3/.style={sibling distance=1.15cm}]
    \node {$A, 0$}
        child {node {$B, 0$}
            child {node {$C, 1$}}
            child {node {$D, 0$}
                child {node {$E, 1$}}
                child {node {$F, 1$}}
            }
        }
        child {node {$G, 1$}
        };
\end{tikzpicture}
\end{center} 
The \textit{Eulerian Tour} of the tree will begin at $A$ and then
track values as makes progress beginning from the left.  This linked
list will look like the following.\\\\ 
\\\\
\\\\
\\\\
\\\\
\\\\
\\\\
HAND DRAW LINKED LIST....Making it in LaTex was not fun...
\\\\
\\\\
\\\\
\\\\
\\\\
\\\\
\\\\
\\\\
\\\\
\\\\
Using the pointer doubling technique, the parallel prefix problem on
the linked list can be solved in \BigO{\log n} time.  Since all the
other steps were doable in \BigO{1} time, the overall runtime of the
algorithm will be \BigO{\log n}.


\section*{Problem 22: }
%Fairly sure that this is the same as 21 except do the Eulerian tour
% + on the left/right and - on the right/left (Depending how the balance
% goes.  Should lead to some runtime of log n/2 which is O(log n)
% all nodes will have to contribute 1 to the sum.

\end{document}
