\documentclass[12pt]{article}

\usepackage{tikz}
\usepackage[ruled, vlined]{algorithm2e}
\usepackage{pst-node,pst-plot}
\usepackage{amsmath}
\usepackage{float}

\newcommand{\BigO}[1]{\ensuremath{\operatorname{\mathcal{O}}\bigl(#1\bigr)}}

\begin{document}
\title{Homework 24}
\author{Robbie McKinstry, Jack McQuown, Cyrus Ramavarapu}
\renewcommand{\today}{27 October 2016}
\renewcommand{\baselinestretch}{1.5}
\maketitle

\section*{Problem 12: }
\section*{Problem 13: }
\section*{Problem 14: }
In order to think of Longest Common Subsequence (LCS) as a graph, we can represent both sequences A and B as a 2D array. Where one index {$i$} will represent the index of a letter in the sequence A, and {$j$} will represent the index of a character in the sequence B.\\
An example of this can be seen in the table below:
\begin{table}[H]
\centering
\caption{LCS Table}
\label{my-label}
\begin{tabular}{ccccccc}
                       & 1 & 5 & 6 & 8 & 2 & 9 \\ \cline{2-7} 
\multicolumn{1}{c|}{0} & 0 & 0 & 0 & 0 & 0 & 0 \\
\multicolumn{1}{c|}{5} & 0 & 1 & 0 & 0 & 0 & 0 \\
\multicolumn{1}{c|}{8} & 0 & 0 & 0 & 1 & 0 & 0 \\
\multicolumn{1}{c|}{2} & 0 & 0 & 0 & 0 & 1 & 0 \\
\multicolumn{1}{c|}{7} & 0 & 0 & 0 & 0 & 0 & 0 \\
\multicolumn{1}{c|}{6} & 0 & 0 & 1 & 0 & 0 & 0
\end{tabular}
\end{table}
This table is filled with both 0's and 1's, where the 0 indicates that {$A[i] \neq B[j]$} and a 1 indicates that {$A[i] \equiv B[j]$}.\\

\end{document}
